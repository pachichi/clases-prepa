\documentclass[handout]{beamer}
\usetheme[]{Szeged}
\usecolortheme{beaver}
\setbeamertemplate{footline}[frame number]
\usepackage[utf8]{inputenc}
\usepackage[spanish,es-nodecimaldot]{babel}
\usepackage{amsmath, amsthm, amsfonts, amssymb}
\usepackage{textcomp}
\usepackage{multimedia}
\DeclareGraphicsExtensions{.png,.pdf,.jpg,.jpeg}
\graphicspath{{imagenes/}} %directorio donde se guardan las imagenes
\usepackage{hyperref}


\title{Física II}
\author{Fis. Sergio Ángel Sánchez Chávez}
\institute[UVM]{4\textdegree \hspace{2pt} cuatrimestre.}
\logo{\includegraphics[width=1.2cm]{uvm}}
\date{\today}

\begin{document}

\begin{frame}[noframenumbering]
  \titlepage
  \begin{center}
    \includegraphics[width=5.5cm]{uvm1}    
  \end{center}  
\end{frame}


\section{Filosofía institucional}

\begin{frame}
  \frametitle{Misión y Visión}
  \begin{center}
    \begin{block}{Misión}
      {\large Ampliamos el acceso a educación de calidad global para formar personas
        productivas que agregan valor a la sociedad.}
    \end{block}
    \begin{block}{Visión}
      {\large Ser la comunidad universitaria privada más influyente en el desarrollo
        sustentable de México.}
    \end{block}
  \end{center}
\end{frame}



\begin{frame}
  \frametitle{Principios}
  \begin{block}{Poder transformador de la Educación.}
  Creemos en la eduación como principio transformador y como derecho de los seres humanos
  a crecer y desarrollarse a través de ella.  
  \end{block}

  \begin{block}{Calidad Académica}
    Creemos en un formación académica de nivel internacional y en nuestra capacidad de
    llevarla a sectores con alto potencial para aprovecharla y convertirla en factor de
    crecimiento personal y movilidad social.
  \end{block}
\end{frame}

\begin{frame}
  \frametitle{Principios}

  \begin{block}{El Estudiante al centro}
    Creemos que el estudiante es el eje del quehacer en la UVM y que mientras mas completa
    sea su experiencia en la Universidad, más sólidas serán sus competencias personales y
    profesionales a partir de las cuales participará en la mejora de su comunidad y la
    sociedad de México y del mundo.
  \end{block}

  \begin{block}{Inclusión}
    Creemos en la pluralidad y la multiculturalidad como signos esenciales de la sociedad,
    por ello estamos convencidos que los criterios incluyentes enriquecen, diversifican y
    abren oportunidades para todos, mientras que las exclusiones empobrecen.   
  \end{block}

\end{frame}

\begin{frame}  
  \frametitle{Principios}

  \begin{block}{Innovación}
    Creemos en nuestra capacidad de creación, diseño e implantación de modalidades y
    escenarios novedosos que nos permitan desarrollarnos de manera orgánica e integrada.
  \end{block}

  \begin{block}{Mejora de procesos}
    Creemos en el mejoramiento permanente como base para optimizar los servición
    educativos y administrativos y sus resultados.
  \end{block}

  \begin{block}{Efectividad}
    Creemos en la importancia de mantener la eficiencia y la eficacia en nuestros procesos
    y servicios, como sello distintivo de nuestra gestión.
  \end{block}
  
\end{frame}


\begin{frame}
  \frametitle{Valores}

  \begin{block}{Integridad en el actuar}
    Realizar con rectitud -honestidad y transparencia- todas nuestras acciones.    
  \end{block}

  \begin{block}{Actitud de servicio}
    Mantener la disposición de ánimo en nuestro actuar y colaborar con los demás, con
    calidez, compromiso, entusiasmo y respeto.
  \end{block}
\end{frame}

\begin{frame}
  \frametitle{Valores}
  
  \begin{block}{Responsabilidad social}
    Asumir con clara conciencia las consecuencias de nuestros actos ante la sociedad.
  \end{block}
  \begin{block}{Cumplimiento de promesas}
    Convertir en compromisos nuestras promesas y asegurar su cumplimiento.
  \end{block}
  \begin{block}{Calidad de ejecución}
    Desempeñar de manera impecable y oportuna las funciones que nos corresponden a partir
    de criterios de excelencia.
  \end{block}
  
\end{frame}

\begin{frame}
  \frametitle{Lema}
  \begin{center}
    \begin{block}
      {\Large Por siempre responsable de lo que se ha cultivado.}
    \end{block}
    
  \end{center}
\end{frame}

\section{Temario}

\begin{frame}
  \frametitle{Física II}
  \begin{block}{}
    \begin{description}
    \item[Bloque I.] Describe los fluidos en reposo y movimiento.
    \item[Bloque II.] Distingue entre calor y temperatura.
    \item[Bloque III.] Comprende las leyes de la electricidad.
    \item[Bloque IV.] Relaciona la electricidad y el magnetismo.
    \end{description}
  \end{block}
\end{frame}


\section{Criterios de evaluación}

\begin{frame}
  \frametitle{Conducta en el aula}
  \begin{itemize}
  \item Portar la playera distintiva de UVM.
  \item Tolerancia para ingresar a clase es de 10 minutos.
  \item No se podrá utilizar el celular, tablet o computadora, a menos que el profesor lo indique.
  \item Prohibido comer dentro del aula.
  \item Para ingresar al laboratorio debera traer su manual y bata.
  \end{itemize}
\end{frame}


\begin{frame}
  \frametitle{Forma de calificar.}
  \begin{description}
  \item[50\%] Examen escrito.
  \item[30\%] Evaluación continua (Trabajo en clase y tareas).
  \item[10\%] Trabajo y asistencia en laboratorio.
  \item[10\%] Reportes de laboratorio.
  \end{description}
\end{frame}

\begin{frame}
  \frametitle{Bibliografía}
  \begin{itemize}
  \item Física 2, Héctor Pérez Montiel, Grupo Editorial Patria.
  \item Física conceptual, Paul G. Hewitt, Addison Wesley.
  \item Física: conceptos y aplicaciones, Paul E. Tippens, McGraw-Hill Interamericana.
  \end{itemize}
\end{frame}
\end{document}