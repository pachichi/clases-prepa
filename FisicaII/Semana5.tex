\documentclass[handout]{beamer}
\usetheme[]{Szeged}
\usecolortheme{beaver}
\setbeamertemplate{footline}[frame number]
\usepackage[utf8]{inputenc}
\usepackage[spanish,es-nodecimaldot]{babel}
\usepackage{amsmath, amsthm, amsfonts, amssymb}
\usepackage{textcomp}
\usepackage{multimedia}
\DeclareGraphicsExtensions{.png,.pdf,.jpg,.jpeg}
\graphicspath{{imagenes/}} %directorio donde se guardan las imagenes
\usepackage{hyperref}
\usepackage{enumitem}

\SetLabelAlign{parright}{\parbox[t]{\labelwidth}{\raggedleft#1}}
\setlist[description]{style=multiline,topsep=10pt,leftmargin=4cm,align=parright}
\setlist[itemize]{label=\textbullet}

\title{Física II}
\author{Ejercicios Dilatación}
\institute[UVM]{4\textdegree \hspace{2pt} cuatrimestre.}
\logo{\includegraphics[width=1.2cm]{uvm}}
\date{\today}

\newcommand{\celsius}{\textdegree~C}
\newcommand{\fahrenheit}{\textdegree~F}
\newcommand{\units}[2]{#1$^{#2}$}
 
\begin{document}

\begin{frame}[noframenumbering]
  \titlepage
  \begin{center}
    \includegraphics[width=5.5cm]{uvm1}    
  \end{center}  
\end{frame}


\begin{frame}[allowframebreaks,t]
  \frametitle{Investigar}
  \begin{block}{ }
    Investigar los coeficientes de dilatación lineal de las siguientes sustancias. Ojo los
    coeficientes deben de estar dados en las unidades de (1/\textdegree C).
    \begin{itemize}
    \item  Hierro
    \item Aluminio
    \item Cobre
    \item Plata
    \item Plomo
    \item Niquel
    \item Acero
    \item Cinc
    \item Vidrio
    \end{itemize}
    \end{block}

    \begin{block}
      
    Investigar los coeficientes de dilatación volumétrica de las siguientes sustancias. Ojo los
    coeficientes deben de estar dados en las unidades de (1/\textdegree C).
    \begin{itemize}
    \item Mercurio
    \item Glicerina
    \item Alcohol etílico
    \item Petróleo
    \end{itemize}
  \end{block}
  

\end{frame}


\begin{frame}[allowframebreaks,t]
  \frametitle{Ejercicios}
  \begin{itemize}
  \item Un puente de acero de 100 m de largo a 8 \textdegree C aumenta su temperatura a 24
    \textdegree C. ¿Cuánto aumentarrá su longitud?
  \item Cuál es la longitud de un riel de hierro a 6 \textdegree C si a 40 \textdegree C
    la longitud es de 50 m. ¿Cuantó se contrae?
  \item A una temperatura de 17 \textdegree una ventana tiene 1.6 m$^2$ ¿Cuál será la
    la superfice final a 32 \textdegree C?
  \item Calcular el volumen final de  5.5 litros de glicerina si se calienta de 4
    \textdegree a 25 \textdegree C. Determinar también la variación de su volumen en
    centímetros cúbicos.
  \end{itemize}
\end{frame}




\begin{frame}[allowframebreaks,t]
  \frametitle{Ejercicios}
  \begin{enumerate}
  \item Un tubo de cobre tiene un volumen de 0.009 \units{m}{3} a 10\celsius y se calienta
    a 200\celsius.
    \begin{itemize}
    \item ¿Cuál es el volumen fin al?
    \item ¿Cuál es su dilatación en cúbica en \units{m}{3} y en litros.
    \end{itemize}
  \item Una barra de alumnio tiene un volumen de 500\units{cm}{3} a 90\celsius.
    \begin{itemize}
    \item ¿Cuál será su volumen a 20\celsius?
    \item ¿Cuánto disminuyo su volumen?
    \end{itemize}
  \item Un tanque de hierro de 200 litros de capacidad a 10\celsius se llena totalmente de
    petróleo si se incrementa la temperatura de ambos hasta 38\celsius.
    \begin{itemize}
    \item ¿Cuál es la dilatación cúbica del tanque?
    \item ¿Cuál es la dilatación cúbica del petróleo?
    \item ¿Cuánto petróleo derramará en litros y centimetros cúbicos?
    \end{itemize}
  \item Un gas a presión constante y a 0\celsius ocupa un volumen de 25 litros. Si su
    temperatura se incrementa a 18\celsius
    \begin{itemize}
    \item ¿Cuál es su volumen final?
    \item ¿Cuál es us dilatación cúbica?
    \end{itemize}

  \end{enumerate}
\end{frame}


\end{document}