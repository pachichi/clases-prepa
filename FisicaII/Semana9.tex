\documentclass[handout]{beamer}
\usetheme[]{Szeged}
\usecolortheme{beaver}
\setbeamertemplate{footline}[frame number]
\usepackage[utf8]{inputenc}
\usepackage[spanish,es-nodecimaldot]{babel}
\usepackage{amsmath, amsthm, amsfonts, amssymb}
\usepackage{textcomp}
\usepackage{multimedia}
\DeclareGraphicsExtensions{.png,.pdf,.jpg,.jpeg}
\graphicspath{{imagenes/}} %directorio donde se guardan las imagenes
\usepackage{hyperref}
\usepackage{enumitem}

\SetLabelAlign{parright}{\parbox[t]{\labelwidth}{\raggedleft#1}}
\setlist[description]{style=multiline,topsep=10pt,leftmargin=4cm,align=parright}
\setlist[itemize]{label=\textbullet}

\title{Física II}
\author{Ejercicios Calor}
\institute[UVM]{4\textdegree \hspace{2pt} cuatrimestre.}
\logo{\includegraphics[width=1.2cm]{uvm}}
\date{\today}

\newcommand{\celsius}{~\textdegree~C}
\newcommand{\fahrenheit}{~\textdegree~F}
\newcommand{\units}[2]{#1$^{#2}$}
 
\begin{document}

\begin{frame}[noframenumbering]
  \titlepage
  \begin{center}
    \includegraphics[width=5.5cm]{uvm1}    
  \end{center}  
\end{frame}


\begin{frame}[allowframebreaks,t]
  \frametitle{Calor y Temperatura}
  \begin{block}{ }
    Lee la secciones del libro \textit{Para tu reflexión}, de las páginas 79, 81 y 91,
    para contestar la siguientes preguntas:
  \end{block}
  \begin{itemize}
  \item ¿Por qué un cuerpo que se encuentra a 0 K no puede transmitir calor?
  \item ¿Cuál es la principal característica de un superconductor?
  \item ¿Cómo mejoraría el tener nuevos materiales superconductores?
  \item ¿Por qué el ecuador es más caliente que los polos?
  \item ¿Cuál es generalmente la dirección del flujo de aire y a qué se debe?
  \item ¿Qué condiciones son necesarias para generar un tornado o huracán?
  \item Haz un dibujo indicando las etapas de cómo se nacen los huracanes en los mares.
  \item Ilustra el proceso de fusión nuclear.
  \item ¿A qué debemos que se atenue la intensidad de energía solar que nosotros recibimos a
    nivel de suelo?
  \item Cierto lugar recibe 7 kilocalorias por minuto. ¿Cuál es la intensidad de energía
    solar que recibe por metro cuadrado?
  \item ¿Qué color es el que absorbe mejor el calor?
  \item Menciona algunas aplicaciones en las que se utilize la energía solar.
  \item Cómo debe ser la gráfica de la densidad del agua respecto 
  \end{itemize}

\end{frame}




\end{document}