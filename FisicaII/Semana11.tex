\documentclass[handout]{beamer}
\usetheme[]{Szeged}
\usecolortheme{beaver}
\setbeamertemplate{footline}[frame number]
\usepackage[utf8]{inputenc}
\usepackage[spanish,es-nodecimaldot]{babel}
\usepackage{amsmath, amsthm, amsfonts, amssymb}
\usepackage{textcomp}
\usepackage{multimedia}
\DeclareGraphicsExtensions{.png,.pdf,.jpg,.jpeg}
\graphicspath{{imagenes/}} %directorio donde se guardan las imagenes
\usepackage{hyperref}
\usepackage{enumitem}

\SetLabelAlign{parright}{\parbox[t]{\labelwidth}{\raggedleft#1}}
\setlist[description]{style=multiline,topsep=10pt,leftmargin=4cm,align=parright}
\setlist[itemize]{label=\textbullet}

\title{Física II}
\author{Bloque 3}
\institute[UVM]{4\textdegree \hspace{2pt} cuatrimestre.}
\logo{\includegraphics[width=1.2cm]{uvm}}
\date{\today}

\newcommand{\celsius}{~\textdegree~C}
\newcommand{\fahrenheit}{~\textdegree~F}
\newcommand{\units}[2]{#1$^{#2}$}
 
\begin{document}

\begin{frame}[noframenumbering]
  \titlepage
  \begin{center}
    \includegraphics[width=5.5cm]{uvm1}    
  \end{center}  
\end{frame}


\begin{frame}
  \frametitle{Trabajo para entregar}
  \begin{block}{}
    Todos los ejercicios bloque 3. En equipos de hasta 4 personas.
  \end{block}
  \begin{itemize}
  \item Ley de Columb. \textbf{10 ejercicios} (pag. 143)
  \item Campo eléctrico.\textbf{8 ejercicios} (pag. 148)
  \item Intensidad de corriente. \textbf{3 ejercicios} (pag. 152)
  \item Resistencia electrica. \textbf{4 ejercicios} (pag. 156)
  \item Ley de Ohm. \textbf{4 ejercicios} (pag. 158)
  \item Resistencia Serie y paralelo. \textbf{5 ejercicios} (pag. 169)
  \item Potencia eléctrica. \textbf{6 ejercicios} (pag. 178)
  \item Ley de Joule. \textbf{5 ejercicios} (pag. 180) 
  \item Capacitores. \textbf{7 ejercicios} (pag. 185) 
  \end{itemize}
\end{frame}



% \section{Electricidad}

% \begin{frame}[allowframebreaks,t]
%   \frametitle{Carga eléctrica}
%   \begin{itemize}
%   \item Carga electrica
%   \item Conservación de la carga
%   \end{itemize}

% \end{frame}

% \begin{frame}
%   \frametitle{Formas de electrizar objetos}
  
% \end{frame}

% \begin{frame}
%   \frametitle{Conductores y Aislantes}
  
% \end{frame}

\section{Electrostática}


\begin{frame}
  \frametitle{Ley de Coulomb}
  \begin{block}{}
    Toda presión que se ejerce sobre un líquido encerrado en un recipiente se transmite
    con la misma intensidad a todos los puntos del líquido y a las paredes del recipiente
    que lo contiene.
  \end{block}
  
  {\huge \[F = k  \frac{q_1 q_2}{r^2}\]}
  
  \begin{tabular}{ll}
    $F:$ & Fuerza (N).  \\ 
    $k:$ & Constante de proporcionalidad (Nm$^2$/C$^2$) \\ 
    $q_i:$ & Carga eléctrica (C). \\
    $r:$ & Distancia entre cargas (m) \\
  \end{tabular}
  
\end{frame}

\section{Electrodinámica}

\begin{frame}
  \frametitle{Intensidad de corriente eléctrica}
  % \begin{block}{}
  %   Corriente electrica
  % \end{block}
  
  {\huge \[I =\frac{q}{t}\]}
  
  \begin{tabular}{ll}
    $I$ & Intensidad de corriente (A).  \\ 
    $q:$ & Carga eléctrica (C) \\ 
    $t:$ & Tiempo (s) \\

  \end{tabular}
  
\end{frame}




\begin{frame}
  \frametitle{Resistencia y Resistividad}
  % \begin{block}{}
  %   Corriente electrica
  % \end{block}
  
  {\huge \[R = \rho \frac{L}{A}\]}
  
  \begin{tabular}{ll}
    $R$ & Resistencia del conductor ($\Omega$).  \\ 
    $\rho:$ & Resistividad del material ($\Omega$m) \\ 
    $L:$ & Longitud (m) \\
    $A:$ & Área (m$^2$) \\
  \end{tabular}
  
\end{frame}

\begin{frame}
  \frametitle{Resistencia y Resistividad}
  % \begin{block}{}
  %   Corriente electrica
  % \end{block}
  
  {\huge \[R_t = R_0 (1+\alpha T)\]}
  
  \begin{tabular}{ll}
    $R_t$ & Resistencia del conductor a cierta temperatura ($\Omega$).  \\ 
    $R_0$ & Resistencia del conductor a 0 \celsius ($\Omega$).  \\ 
    $\alpha:$ & Coef. de temperatura de resistencia del material (1 / \celsius). \\ 
    $T:$ & Temperatura (\celsius). \\
  \end{tabular}
  
\end{frame}


\begin{frame}
  \frametitle{Ley de Ohm}
  % \begin{block}{}
  %   Corriente electrica
  % \end{block}
  
  {\huge \[I  = \frac{V}{R}\]}
  
  \begin{tabular}{ll}
    $I:$ & Intensidad de corriente (A).  \\ 
    $V:$ & Diferencia de potencial (V).  \\ 
    $R:$ & Resistencia del conductor ($\Omega$). \\ 
  \end{tabular}
  
\end{frame}


\begin{frame}
  \frametitle{Resistencia en serie y paralelo}
  \begin{block}{Conectadas en serie}
    {\huge \[R_e = R_1 + R_2 + \cdots +R\]}
  \end{block}

  \begin{block}{Conectadas en paralelo}
    {\huge \[\frac{1}{R_e} = \frac{1}{R_1}+\frac{1}{R_2}+ \cdots + \frac{1}{R_N}\]}
  \end{block}

\end{frame}



\begin{frame}
  \frametitle{Potencia eléctrica }
  \begin{block}{Conectadas en serie}
    {\huge \[P = VI\]}
  \end{block}

  \begin{tabular}{ll}
    $P:$ & Potencia eléctrica (W).  \\ 
    $I:$ & Intensidad de corriente (A).  \\
    $V:$ & Diferencia de potencial (V).  \\ 
  \end{tabular}

\end{frame}


\begin{frame}
  \frametitle{Ley de Joule}
  \begin{block}{Conectadas en serie}
    {\huge \[Q =  0.24I^2Rt\]}
  \end{block}

  \begin{tabular}{ll}
    $Q:$ & Calor generado (J).  \\ 
    $I:$ & Intensidad de corriente (A).  \\ 
    $R:$ & Resistencia del conductor ($\Omega$). \\ 
    $t:$ & Tiempo del aparato conectado (s). \\ 
  \end{tabular}

\end{frame}



\begin{frame}
  \frametitle{Capacitancia}
  \begin{itemize}
  \item ¿Qué es un capacitor?
  \item ¿Cuál es la función del capacitor?
  \item ¿Cuál es la ecuación que relaciona la capacitancia y el voltaje?
  \item ¿En que unidades se mide la capacitancia?
  \item ¿Cómo se calcula la capacitancia total de capacitores en serie y en paralelo?
  \item Dibuja el interior de un capacitor.
  \end{itemize}
\end{frame}

\end{document}