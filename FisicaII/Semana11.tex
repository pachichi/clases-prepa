\documentclass[handout]{beamer}
\usetheme[]{Szeged}
\usecolortheme{beaver}
\setbeamertemplate{footline}[frame number]
\usepackage[utf8]{inputenc}
\usepackage[spanish,es-nodecimaldot]{babel}
\usepackage{amsmath, amsthm, amsfonts, amssymb}
\usepackage{textcomp}
\usepackage{multimedia}
\DeclareGraphicsExtensions{.png,.pdf,.jpg,.jpeg}
\graphicspath{{imagenes/}} %directorio donde se guardan las imagenes
\usepackage{hyperref}
\usepackage{enumitem}

\SetLabelAlign{parright}{\parbox[t]{\labelwidth}{\raggedleft#1}}
\setlist[description]{style=multiline,topsep=10pt,leftmargin=4cm,align=parright}
\setlist[itemize]{label=\textbullet}

\title{Física II}
\author{Bloque 3}
\institute[UVM]{4\textdegree \hspace{2pt} cuatrimestre.}
\logo{\includegraphics[width=1.2cm]{uvm}}
\date{\today}

\newcommand{\celsius}{~\textdegree~C}
\newcommand{\fahrenheit}{~\textdegree~F}
\newcommand{\units}[2]{#1$^{#2}$}
 
\begin{document}

\begin{frame}[noframenumbering]
  \titlepage
  \begin{center}
    \includegraphics[width=5.5cm]{uvm1}    
  \end{center}  
\end{frame}


\begin{frame}
  \frametitle{Trabajo para entregar}
  \begin{block}{}
    Todos los ejercicios bloque 3. En equipos de hasta 4 personas.
  \end{block}
  \begin{itemize}
  \item Ley de Columb. \textbf{10 ejercicios} (pag. 143)
  \item Campo eléctrico.\textbf{8 ejercicios} (pag. 148)
  \item Intensidad de corriente. \textbf{3 ejercicios} (pag. 152)
  \item Resistencia electrica. \textbf{4 ejercicios} (pag. 156)
  \item Ley de Ohm. \textbf{4 ejercicios} (pag. 158)
  \item Resistencia Serie y paralelo. \textbf{5 ejercicios} (pag. 169)
  \item Potencia eléctrica. \textbf{6 ejercicios} (pag. 178)
  \item Ley de Joule. \textbf{5 ejercicios} (pag. 180) 
  \item Capacitores. \textbf{7 ejercicios} (pag. 185) 
  \end{itemize}
\end{frame}



\section{Electricidad}

\begin{frame}[allowframebreaks,t]
  \frametitle{Carga eléctrica}
  \begin{itemize}
  \item Carga electrica
  \item Conservación de la carga
  \end{itemize}

\end{frame}

\begin{frame}
  \frametitle{Formas de electrizar objetos}
  
\end{frame}

\begin{frame}
  \frametitle{Conductores y Aislantes}
  
\end{frame}

\section{Electrostática}


\begin{frame}
  \frametitle{Ley de Coulomb}
  \begin{block}{}
    Toda presión que se ejerce sobre un líquido encerrado en un recipiente se transmite
    con la misma intensidad a todos los puntos del líquido y a las paredes del recipiente
    que lo contiene.
  \end{block}
  
  {\huge \[\frac{F}{A} = \frac{f}{a}\]}
  
  \begin{tabular}{ll}
    $F:$ & Fuerza sobre émbolo mayor (N).  \\ 
    $A:$ & Área del émbolo mayor (m$^2$) \\ 
    $f:$ & Fuerza sobre émbolo menor (N). \\
    $a:$ & Área del émbolo menor (m$^2$) \\
  \end{tabular}
  
\end{frame}

\section{Electrodinámica}


\end{document}