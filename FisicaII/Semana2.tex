\documentclass[handout]{beamer}
\usetheme[]{Szeged}
\usecolortheme{beaver}
\setbeamertemplate{footline}[frame number]
\usepackage[utf8]{inputenc}
\usepackage[spanish,es-nodecimaldot]{babel}
\usepackage{amsmath, amsthm, amsfonts, amssymb}
\usepackage{textcomp}
\usepackage{multimedia}
\DeclareGraphicsExtensions{.png,.pdf,.jpg,.jpeg}
\graphicspath{{imagenes/}} %directorio donde se guardan las imagenes
\usepackage{hyperref}
\usepackage{enumitem}

\SetLabelAlign{parright}{\parbox[t]{\labelwidth}{\raggedleft#1}}
\setlist[description]{style=multiline,topsep=10pt,leftmargin=4cm,align=parright}
\setlist[itemize]{label=\textbullet}

\title{Física II}
\author{Fluidos}
\institute[UVM]{4\textdegree \hspace{2pt} cuatrimestre.}
\logo{\includegraphics[width=1.2cm]{uvm}}
% \date{\today}

\begin{document}

\begin{frame}[noframenumbering]
  \titlepage
  \begin{center}
    \includegraphics[width=5.5cm]{uvm1}    
  \end{center}  
\end{frame}


\begin{frame}
  \frametitle{Principio de Pascal}
  \begin{block}{}
    Toda presión que se ejerce sobre un líquido encerrado en un recipiente se transmite
    con la misma intensidad a todos los puntos del líquido y a las paredes del recipiente
    que lo contiene.
  \end{block}
  
  {\huge \[\frac{F}{A} = \frac{f}{a}\]}
  
  \begin{tabular}{ll}
    $F:$ & Fuerza sobre émbolo mayor (N).  \\ 
    $A:$ & Área del émbolo mayor (m$^2$) \\ 
    $f:$ & Fuerza sobre émbolo menor (N). \\
    $a:$ & Área del émbolo menor (m$^2$) \\
  \end{tabular}
\end{frame}



\begin{frame}
  \frametitle{Principio de Arquímedes}
  \begin{block}{}
    La magnitud de la fuerza debida al empuje ejercido por un fluido cuando un objeto se
    sumerge en él depende de peso específico y del volumen del fluido desalojado.   
  \end{block}
  {\huge \[E = P_{e}V\]}
  
  \begin{tabular}{ll}
    $E:$ & Empuje sobre el objeto (N).  \\ 
    $P_{e}:$ & Peso específico del líquido (N/m$^3$). \\ 
    $V:$ & El volumen del objeto sumergido en el líquido (m$^3$). \\
  \end{tabular}
\end{frame}



\begin{frame}
\frametitle{Ejercicios resueltos}
\begin{enumerate}
\item Sobre una prensa hidráulica se ejerce sobre el émbolo menor una fuerza de 79 N. Si
  el área del émbolo mayor es 38 cm$^2$ y el área del émbolo menor 12 cm$^2$, calcular la
  fuerza que se obtiene en el émbolo mayor.
\item El radio de un émbolo menor es de 0.2 m. Si se aplica una fuerza de 244 N sobre el
  émbolo mayor y se obtiene una fuerza de 2531 N sobre el émbolo menor, cuál es el radio
  del émbolo mayor. 
\item Un cubo de plomo de con un volumen de 25 cm$^3$ se sumerge completamente dentro de
  de aceite. Calcular el empuje que recibe el cubo. Cuál es el peso aparente del cubo.
\item Una esfera de madera, con radio 0.1 m, se sumerge hasta la mitad en agua de mar, ¿cuál es empuje
  recibido? ¿Cuál es el peso aparente de la esfera?
\item ¿Puede el acero flotar en el agua?
\end{enumerate}
\end{frame}

\begin{frame}[allowframebreaks,t]
\frametitle{Ejercicios propuestos}
\begin{enumerate}
\item Calcular la magnitud de la guerza que se aplica en el émbolo menor de una prensa
  hidráulica de 10 cm$^2$ de área, si en el émbolo mayor con una área de 150 cm$^2$ se
  produce una fuerza  cuya magnitud es 10500N. 
\item Una fuerza de 400 N se aplica al pistón pequeño de una prensa hidráulica cuyo
  diámetro es 4 cm. ¿Cuál deberá ser el diámetro del pistón grande para que pueda levantar
  una carga de 200 kg?

\item Un trozo de hielo grande flota en vaso de agua, de modo que el nivel de ésta queda
  hasta el borde del vaso. ¿Se derramará agua cuando el hielo se derrita? Explicar.

\item Un cubo de acero de 18 cm de arista se sumerge totalmente en agua. Si la magnitud
  de su peso es de 480 N, calcula:
  \begin{itemize}
  \item ¿qué magnitud de empuje recibe?
  \item ¿cuál será la magnitud de peso aparente del cubo?
  \end{itemize}
  
\item Un prisma rectangular de cobre, de base igual a 36 cm$^2$ y una altura de 10 cm, se
  sumerge hasta la mitad por medio de un alambre en un recipiente que contiene
  alcohol. $\rho_{alcohol} = 790 $ kg/m$^3$
  \begin{itemize}
  \item ¿Qué volumen de alcohol desaloja?
  \item ¿Qué magnitud de empuje recibe?
  \item ¿Cuál es el peso aparente del prisma debido al empuje, si su peso real es de 31.36
    N?
  \end{itemize}

\item Un globo meteorológico requiere operar a una altitud donde la densidad del aire es
  0.9 kg/m$^3$. A esa altitud, el globo tiene un volumen de 20 m$^3$ y está lleno de helio
  ($\rho_{He} = 0.178$) kg/m$^3$. Si la bolsa del globo pesa 88 N, ¿qué carga es capaz de
  soportar a este nivel?
\end{enumerate}
\end{frame}


\end{document}