\documentclass[handout]{beamer}
\usetheme[]{Szeged}
\usecolortheme{beaver}
\setbeamertemplate{footline}[frame number]
\usepackage[utf8]{inputenc}
\usepackage[spanish,es-nodecimaldot]{babel}
\usepackage{amsmath, amsthm, amsfonts, amssymb}
\usepackage{textcomp}
\usepackage{multimedia}
\DeclareGraphicsExtensions{.png,.pdf,.jpg,.jpeg}
\graphicspath{{imagenes/}} %directorio donde se guardan las imagenes
\usepackage{hyperref}
\usepackage{enumitem}

\SetLabelAlign{parright}{\parbox[t]{\labelwidth}{\raggedleft#1}}
\setlist[description]{style=multiline,topsep=10pt,leftmargin=4cm,align=parright}
\setlist[itemize]{label=\textbullet}

\title{Física II}
\author{Ejercicios Calor}
\institute[UVM]{4\textdegree \hspace{2pt} cuatrimestre.}
\logo{\includegraphics[width=1.2cm]{uvm}}
\date{\today}

\newcommand{\celsius}{~\textdegree~C}
\newcommand{\fahrenheit}{~\textdegree~F}
\newcommand{\units}[2]{#1$^{#2}$}
 
\begin{document}

\begin{frame}[noframenumbering]
  \titlepage
  \begin{center}
    \includegraphics[width=5.5cm]{uvm1}    
  \end{center}  
\end{frame}


\begin{frame}[allowframebreaks,t]
  \frametitle{Investigar}
  \begin{block}{ }
    Investigar los calores específicos \textit{Ce} (a presión constante) de las siguientes
    sustancias. Darlos en unidades de cal/g \celsius y en unidades de J/kg \celsius. 
    \begin{itemize}
    \item Agua
    \item Hielo
    \item Vapor
    \item Hierro
    \item Cobre
    \item Aluminio
    \item Plata
    \item Vidrio
    \item Mercurio
    \item Plomo
    \end{itemize}
    \end{block}

\end{frame}


\begin{frame}[allowframebreaks,t]
  \frametitle{Ejercicios de calor}
  \begin{itemize}
  \item ¿Qué cantidad de calor se debe aplicar a un cubo de plomo de 850 g para que eleve
    sus temperatura de 18 \celsius a 120 \celsius? \textit{Ce}$_{Pb}$ = 0.031 cal/g \celsius
  \item La temperatura inicial de una placa de alumnio de 3 kg es de 25 \celsius ¿Cuál
    será su temperatura final si al ser calentada recibe 12000 calorias?
    \textit{Ce}$_{Pb}$ = 0.217 cal/g \celsius
  \item ¿Qué cantidad de calor necesitan 60 ml de agua cuya masa es de 60 g para que su
    temperatura aumente de 25 \celsius a 100 \celsius?
  \item Determina las calorías requeridas por una lamina de cobre de 2.5 kg para que su
    temperatura aumente de 12 \celsius a 300 \celsius?
  \item Determina el calor específico de un trozo de metal de 400 g si al suministrarle
    620 calorías aumentó su temperatura de 15 \celsius a 65 \celsius. ¿De qué sustancia se trata?
  \item Dos litros de agua, cuya masa es de 2 kg de agua se enfrían de 100 \celsius a 15
    \celsius. ¿Qué cantidad de calor al ambiente?
  \end{itemize}
\end{frame}


\begin{frame}[allowframebreaks,t]
  \frametitle{Ejercicios de transferencia de calor}
  \begin{itemize}
  \item Se tienen 1000 g de agua a 90 \celsius se combinan con 1000 g de agua a 60
    \celsius. Calcular la temperatura final de la solución.
  \item Determina la temperatura a la que se calentó una bara de hierro de 3 kg si al ser
    introducido en 2 kg de agua a 15 \celsius eleva la temperatura de ésta hasta 30 \celsius.
  \item Un recipiente de aluminio de 150 g contiene 200 g de agua a 10
    \celsius. Determinar la temperatura final del recipiente y del agua, si se introduce
    en ésta un trozo de cobre de 60 g a una temperatura de 300 \celsius.
  \item Una barra de plata de 335.2 g con una temperatura de 100 \celsius se introduce un
    calorímetro de aluminio de  60 g de masa que contiene  450 g de agua a 23 \celsius. Se
    agita la mezcla y la temperatura se incrementa hasta 26 \celsius. ¿Cuál es el calor
    específico de la plata?
  \end{itemize}
  
\end{frame}




\end{document}